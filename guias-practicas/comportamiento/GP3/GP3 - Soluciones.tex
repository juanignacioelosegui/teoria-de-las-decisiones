\documentclass{article}
\usepackage{graphicx, soul}
\usepackage[dvipsnames]{xcolor}

\setul{0.5ex}{0.3ex}

\newcommand{\ulcolor}[2][Red]{\setulcolor{#1}\ul{#2}}
\newcommand*\sepline{%
  \begin{center}
    \rule[1ex]{.5\textwidth}{.5pt}
  \end{center}}

\title{Guía Práctica 3 - Comportamiento (Resuelto)}
\author{Juani Elosegui}
\date{Noviembre 2024}

\begin{document}
    
    \maketitle

    \section*{\underline{Ejercicio 1}}
        \textbf{Datos:}
        \begin{itemize}
            \item Apuesta 1:
            
            Opción A: Obtener \$1.000.000 seguro. 
            
            Opción B: Obtener \$1.000.000 con 98\% de probabilidad, \$2.500.000 con 1\% de probabilidad ó \$0 con 1\% de probabilidad.
            \item Apuesta 2:

            Opción A: Obtener \$1.000.000 con 2\% de probabilidad o \$0 con 98\% de probabilidad.

            Opción B: Obtener \$2.500.000 con 1\% de probabilidad o \$0 con 99\% de probabilidad.
        \end{itemize}
        \textbf{Demuestre que si alguien elige A en la apuesta 1 y B en la apuesta 2, entonces no existe ninguna función de utilidad que pueda explicar sus preferencias mediante EUT}
        \\
        \\
        El bienestar de la gente es algo totalmente subjetivo -o, técnicamente, algo logarítmico- ya que depende en cómo cambia la riqueza de las personas.
        \\
        Podemos ver que en la primera apuesta, la gente suele fatalizar la probabilidad de que no reciba dinero en absoluto. Parece ser que ese 1\% suena como algo terrible para ellos. Esto sucede porque la gente sobreestima las probabilidades bajas, y subestima las probabilidades altas. Estas probabilidades tienen un peso psicológico que distorsiona su decisión. Por eso, se espera que elija la opción A, donde se obtiene \$1.000.000 seguro.
        \\
        \\
        La teoría de la utilidad esperada establece que los individuos eligen la opción que maximiza la utilidad esperada. Esto se expresa como: \(\sum_{i}p_{i} \cdot u(x_{i})\). Si calculásemos las utilidades esperadas de las opciones que elige de la apuesta 1 tenemos que:
        \[EUT_{1A}:u(1.000.000)\]
        \[EUT_{1B}:0,98 \cdot u(1.000.000) + 0,01 \cdot u(2.500.000) + 0,01 \cdot u(0)\]
        Si calculamos la EUT de la apuesta 2:
        \[EUT_{2A}: 0,02 \cdot u(1.000.000) + 0,98 \cdot u(0)\]
        \[EUT_{2B}: 0,01 \cdot u(2.500.000) + 0,99 \cdot u(0)\]
        \textit{COMENTARIO: No conocemos la función de utilidad, por eso directamente la denotamos como $u(.)$}
        \\
        \\
        Si sabemos que una persona elige A en la apuesta 1, quiere decir que:
        \[EUT_{1A} > EUT_{1B}\]
        \[u(1.000.000) > 0,98 \cdot u(1.000.000) + 0,01 \cdot u(2.500.000) + 0,01 \cdot u(0)\]
        \[0,02 \cdot u(1.000.000) > 0,01 \cdot u(2.500.000) + 0,01 \cdot u(0)\]
        \\
        Si sabemos también que una persona elige B en la apuesta 2, quiere decir que:
        \[EUT_{2B} > EUT_{2A}\]
        \[0,01 \cdot u(2.500.000) + 0,99 \cdot u(0) > 0,02 \cdot u(1.000.000) + 0,98 \cdot u(0)\]
        \[0,01 \cdot u(2.500.000) + 0,01 \cdot u(0) > 0,02 \cdot u(1.000.000)\]
        \\
        De las desigualdades despejadas, no se pueden cumplir las dos en simultáneo, lo que nos lleva a una contradicción. Esto demuestra que, cuando se habla de probabilidades o sesgos psicológicos, las funciones de utilidad a veces dejan de tomar papeles relevantes.
        \\
        \\
        Se justifica bien con la \emph{paradoja de Allais}.

    \section*{\underline{Ejercicio 2}}
        \textbf{Datos:}
        \begin{itemize}
            \item Se tienen 12 bolitas: 4 son \ulcolor[blue]{azules} y las 8 restantes son \ulcolor[green]{verdes} o \ulcolor[purple]{violetas}.
            \item Se saca al azar una bolita de esta urna y se presentan las dos apuestas:
            \item Apuesta 1:

            Opción A: ganar \$1.000.000 si la bolita es \ulcolor[blue]{azul}.

            Opción B: ganar \$1.000.000 si la bolita es \ulcolor[green]{verde}.
            \item Apuesta 2:

            Opción A: ganar \$1.000.000 si la bolita es \ulcolor[blue]{azul} o \ulcolor[purple]{violeta}.

            Opción B: ganar \$1.000.000 si la bolita es \ulcolor[green]{verde} o \ulcolor[purple]{violeta}.
        \end{itemize}
        \textbf{Demuestre que si alguien elige A en la apuesta 1 y B en la apuesta 2, entonces no existe ninguna función de utilidad que pueda explicar sus preferencias mediante EUT}
        \\
        \\
        Este inciso se puede justificar muy bien con la \emph{paradoja de Ellsberg}. La gente no parece pesar sus decisiones por la probabilidad de ocurrencia de los eventos (no calcula utilidades esperadas).
        \\
        \\
        Calculemos qué fue lo que pasó para que veamos que, en efecto, sí sucede una paradoja.
        \\
        La teoría de la utilidad esperada establece que los individuos eligen la opción que maximiza la utilidad esperada. Esto se expresa como: \(\sum_{i}p_{i} \cdot u(x_{i})\). Si calculásemos las utilidades esperadas de las opciones que elige de la apuesta 1 tenemos que:
        \[EUT_{1A}: \frac{4}{12} \cdot u(1.000.000)\]
        \[EUT_{1B}: \frac{8 - x_{violetas}}{12} \cdot u(1.000.000)\]
        Si calculamos la EUT de la apuesta 2:
        \[EUT_{2A}: \frac{4}{12} \cdot u(1.000.000) + \frac{8-y_{verdes}}{12} \cdot u(1.000.000)\]
        \[EUT_{2B}: \frac{8-x_{violetas}}{12} \cdot u(1.000.000) + \frac{8-y_{verdes}}{12} \cdot u(1.000.000)\]
        Si sabemos que eligió dos veces la opción A, no existe ninguna función de utilidad posible que satisfaga las condiciones.
        \\
        \\
        \textit{\ulcolor[red]{Me debería esmerar en hacer alguna explicación con cuentas más fácil. De todas formas, no entendí mucho el PDF del pizarrón. Preguntarle a Mathi o a alguien más.}}

\sepline

    \section*{\underline{Ejercicio 1}}
        \textbf{Datos:}
        \begin{itemize}
            \item Una persona recibe un cheque de \$1.900, y se le presentan dos opciones:
            
            Opción A: terminar con \$1.950 seguro.

            Opción B: terminar con \$1.800 con una probabilidad del 20\% o terminar con \$2.000 con una probabilidad del 80\%.
        \end{itemize}
        \textbf{Calcule el valor monetario esperado de cada opción y la decisión óptima según EMVT}
        \\
        \\
        \[EMV_{A}: 1950\]
        \[EMV_{B}: 0,2 \cdot 1.800 + 0,8 \cdot 2.000\]
        \[EMV_{B}: 1.960\]
        Como $EMV_{B} > EMV_{A}$, la decisión óptima según EMVT es elegir la opción B.
        \\
        \\
        \textbf{Calcule la utilidad esperada de cada opción y la decisión óptima según EUT para un agente cuya función de utilidad es \(u(v)=\log_{10} (v+1)\)}
        \\
        \\
        \[EU_{A}: u(1950) = \log_{10}(1.950 + 1)\]
        \[EU_{A}: u(1.950) = \log_{10}(1.951)\]
        \[EU_{A}: u(1.950) \approx 3,2902\]
        \\
        \[EU_{B}: u(.) = 0,2 \cdot \log_{10}(1.800 + 1) + 0,8 \cdot \log_{10}(2.000 + 1)\]
        \[EU_{B}: u(.) = 0,2 \cdot \log_{10}(1.801) + 0,8 \cdot \log_{10}(2.001)\]
        \[EU_{B}: u(.) \approx 3,2921\]
        \\
        Como \(EU_{B} > EU_{A}\), la decisión óptima sigue siendo la opción B.
        \\
        \\
        \textbf{Calcule el valor psicológico esperado de cada opción y la decisión predicha según PT asumiendo que la función de utilidad tiene una raíz quinta para las ganancias, una raíz cuadrada para las pérdidas, y que la función de peso decisional es \(\pi(p) = \frac{1}{e^{(ln(\frac{1}{p})^{0,7})}}\)}
        \\
        \\
        \[v(x) = \left\{\begin{array}{lcc} x^{\frac{1}{5}} & si & R \geq V \\ \\ -|x|^{\frac{1}{2}} & si & R < V\end{array} \right.\]
        La rama de arriba es la ganancia; la de abajo, las pérdidas.
        \\
        \\
        Los valores psicológicos (PV) son:
        \[PV_{A}: v(1.950)\]
        ¿Qué función usamos? Este es un evento probable -seguro, por cierto-, por lo que vamos a usar la función de las ganancias:
        \[PV_{A}: v(1.950) = 1.950^{\frac{1}{5}}\]
        \[PV_{A}: v(1.950) \approx 4,55\]
        \\
        \[PV_{B}: \pi(0,2) \cdot v(1.800) + \pi(0,8) \cdot v(2.000)\]
        \[PV_{B} \approx 4,92\]
        \textit{COMENTARIO: Notar que no usamos la rama de las pérdidas, ya que todas son interpretadas como ganancias netas.}
        \\
        \\
        \textbf{¿Cambia la respuesta al punto anterior si la persona recibía un cheque de \$2.000?}
        \\
        \\
        Si recibimos un cheque de \$2.000, y después nos dicen que nos van a terminar dando \$1.950; no sólo nos cagaron, si no que perdimos \$50. Ahora sí vamos a usar rama de las pérdidas.
        \\
        \[PV_{A}: v(-50) = -|-50|^{\frac{1}{2}}\]
        \[PV_{A}: v(-50) \approx 7,07\]
        \\
        \[PV_{B}: v(-200), v(-0) = \pi(0,2) \cdot v(-200) + \pi(0,8) \cdot v(-0)\]
        \[PV_{B}: v(-200), v(-0) \approx -3,45\]
        Conclusión: sí cambia. Nunca fui de muchas palabras.
    \section*{Ejercicio 2}
        \textbf{Datos:}
        \begin{itemize}
            \item Pedro interpreta a las pérdidas como una \(x^{\frac{1}{2}}\) y a las ganancias como una \(x^{\frac{1}{5}}\).
            \item Su función de pesos decisionales es \(\pi(p) = \frac{1}{e^{(ln(\frac{1}{p})^{0,7})}}\).
            \item Situación A)

            Víctor le regala a Pedro \$100. Después, lo invita a apostar: gana \$100 más con total seguridad como opción 1, o gana \$200 más con una probabilidad del 80\% como opción 2.

            \item Situación B)

            Víctor le regala \$300. Después, lo invita a apostar: pierde \$200 con total seguridad como opción 1, o pierde los \$300 con una probabilidad del 20\% como opción 2.
        \end{itemize}
        \textbf{¿Qué elige Pedro en cada escenario?}
        \\
        \\
        Calculo el valor psicológico de las dos opciones en la situación a):
        \[PV_{A1}: v(100 + 100)\]
        \[PV_{A1}: v(200)\]
        \[PV_{A1}: 200^{\frac{1}{5}}\]
        \textit{COMENTARIO: Notar que no tenemos pérdidas, sólo usamos la rama de la función partida para las ganancias netas}.
        \[PV_{A1} \approx 2,891\]
        \\
        \[PV_{A2}: \pi(0,8) \cdot v(100 + 200)\]
        \[PV_{A2}: \pi(0,8) \cdot v(300)\]
        \[PV_{A2}: \frac{1}{e^{(ln(\frac{1}{0,8})^{0,7})}} \cdot 300^{\frac{1}{5}}\]
        \[PV_{A2} \approx 3,061\]
        Como \(PV_{A2} > PV_{A1}\), Pedro optará por la opción 1.
        \\
        \textit{COMENTARIO: si fueran iguales en términos aproximados de valor psicológico, elige cualquiera de las dos. Es una pelotudez lo que estoy diciendo, pero es un detalle que me di cuenta.}
        \\
        \\
        Calculo el valor psicológico de las dos opciones en la situación b):
        \[PV_{B1}: v(300 - 200)\]
        \[PV_{B1}: v(100)\]
        \[PV_{B1}: 100^{\frac{1}{5}}\]
        \[PV_{B1} \approx 2,51\]
        \\
        \[PV_{B2}: \pi(0,2) \cdot v(300 - 300)\]
        \[PV_{B2}: \frac{1}{e^{(ln(\frac{1}{0,2})^{0,7})}} \cdot v(0)\]
        \[PV_{B2}: \frac{1}{e^{(ln(\frac{1}{0,2})^{0,7})}} \cdot 0^{\frac{1}{5}}\]
        \[PV_{B2} = 0\]
        Pedro elige la opción 1 también.
        \\
        \textit{\ulcolor[red]{Preguntar. Tengo más dudas en este último que Ricky Martin a los 15.}}
        \\
        \\
        Por lo tanto, la opción correcta es la A) Elige la opción 1 en ambas situaciones.

\end{document}
