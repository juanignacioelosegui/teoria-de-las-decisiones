\documentclass{article}
\usepackage{graphicx}

\title{Guía Práctica 1 - Comportamiento (Resuelto)}
\author{Juani Elosegui}
\date{Octubre 2024}

\begin{document}
    \maketitle

    \section*{\underline{Ejercicio 1}}
        No quiere decir necesariamente que Homero Simpson sea un pelotudo, simplemente que su comportamiento no es el esperado teniendo como referencia a un \textit{econ}.

    \section*{\underline{Ejercicio 2}}
        \textit{El sistema 1 es aquel que es rápido, instintivo, emocional y subconsciente. El sistema 2 es aquel que es lento, lógico y consciente.}

        \begin{itemize}
            \item Sistema 1
            \item Sistema 2
            \item Sistema 2
            \item Sistema 1
            \item Sistema 2
            \item Sistema 1
            \item Sistema 1
        \end{itemize}

    \section*{\underline{Ejercicio 3}}
        Retiraría la trucha (porque no es salmón), y la hamburguesa (porque no es exclusivo). Dejaría el bife de lomo, el pollo teriyaki (porque son exclusivos), la ensalada césar (porque puede ser un acompañamiento para el salmón) y el salmón.

    \section*{\underline{Ejercicio 4}}
        El efecto señuelo es efectivo cuando se elige entre productos con al menos dos atributos porque necesita que las opciones difieran en más de un aspecto para que el señuelo pueda influir en la percepción comparativa.

    \section*{\underline{Ejercicio 5}}
        Son suficientes las siguientes afirmaciones para decir que "A domina parcialmente a B":
        \begin{itemize}
            \item A es mejor que B en un atributo
            \item A y B tienen al menos dos atributos.
        \end{itemize}
        Son suficientes las siguentes afirmaciones para decir que "B domina totalmente a A":
        \begin{itemize}
            \item A y B tienen dos atributos.
            \item B es mejor que A en dos atributos.
        \end{itemize}

    \section*{\underline{Ejercicio 6}}
        No se cumplen las siguientes afirmaciones:
        \begin{itemize}
            \item Los agentes tienen expectativas racionales.
            \item Los agentes son indiferentes ante alternativas irrelevantes.
            \item Los agentes tienen una función de utilidad bien definida.
            \item Los agentes prefieren la mayor cantidad de alternativas posibles.
        \end{itemize}

\end{document}
